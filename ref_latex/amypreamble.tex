\documentclass[10pt, a4paper]{article} 
\usepackage{fontspec} % Font selection for XeLaTeX; see fontspec.pdf. 
\usepackage{titlesec,titletoc}
\usepackage[AutoFakeBold=true, AutoFakeSlant=true]{xeCJK}	% 中文使用 XeCJK,利用 \setCJKmainfont 定義中文內文、粗體與斜體的字型	
\usepackage{zhnumber}
%\usepackage[BoldFont, SlantFont]{xeCJK}% 中文使用 XeCJK,並模擬粗體與斜體(\textbf{ } \textit{ })
\defaultfontfeatures{Mapping=tex-text} % to support TeX conventions like ``---''
\usepackage{xunicode} % Unicode support for LaTeX character names(accents, European chars, etc)
\usepackage{xltxtra} 						% Extra customizations for XeLaTeX
\usepackage{amsmath, amssymb}
%\usepackage{enumitem}
%\usepackage{graphicx, subfig, float, wrapfig} % support the \includegraphics command and options
%\usepackage[outercaption]{sidecap} %[options]=[outercaption], [innercaption], [leftcaption], [rightcaption]
\usepackage{array, booktabs}
\usepackage{color, xcolor}
\usepackage{longtable}
\usepackage{colortbl}
\usepackage{wrapfig}
\usepackage{subfig}
\usepackage{graphicx}                          				
\usepackage{listings}						%直接將 latex 碼轉換成顯示文字
\usepackage{hyperref}
%\usepackage[parfill]{parskip} 				% 新段落前加一空行,不使用縮排
%{}裡的套件名稱,[]裡面是指令
%\usepackage{geometry} 
\usepackage[left=2.5cm,right=2.5cm,top=1.6cm,bottom=2.54cm]{geometry} 
\usepackage{multicol}
\usepackage{multirow}
\usepackage{tabularx}
\usepackage{indentfirst}
\usepackage{natbib}
\usepackage{ulem}
\usepackage{CJKnumb}
\usepackage{xpatch}
\usepackage{appendix}
\usepackage{dcolumn}
\usepackage{longtable}	

% XeCJK package:
%-----------------------------------------------------------------
%  中英文內文字型設定
\setCJKmainfont{Microsoft JhengHei}
% 設定中文內文字型
% 定義粗體的字型(依使用的電腦安裝的字型而定)				
\setmainfont{Times New Roman}				
% 設定英文內文字型
\setsansfont{Arial}						
% 無襯字字型 used with {\sffamily ...}
\setmonofont{Courier New}				
% 等寬字型 used with {\ttfamily ...}
%\setmonofont[Scale=MatchLowercase]{Andale Mono}
% 其他字型(隨使用的電腦安裝的字型不同,用註解的方式調整(打開或關閉))
% 英文字型
\newfontfamily{\E}{Cambria}				
\newfontfamily{\A}{Arial}
\newfontfamily{\Cambria}[Scale=0.9]{Cambria}
\newfontfamily{\TNR}{Times New Roman}
\newfontfamily{\TT}[Scale=0.8]{Times New Roman}
% 中文字型
\newCJKfontfamily{\MB}{Microsoft JhengHei}				% 適用在 Mac 與 Win
\newCJKfontfamily{\SM}[Scale=0.8]{Microsoft JhengHei}	% 縮小版新細明體
\newCJKfontfamily{\K}{DFKai-SB}                	% Windows下的標楷體
% 以下為自行安裝的字型:CwTex 組合
%\newCJKfontfamily{\CF}{cwTeX Q Fangsong Medium}	% CwTex 仿宋體
%\newCJKfontfamily{\CB}{cwTeX Q Hei Bold}			% CwTex 粗黑體
%\newCJKfontfamily{\CK}{cwTeX Q Kai Medium}   		% CwTex 楷體
%\newCJKfontfamily{\CM}{cwTeX Q Ming Medium}		% CwTex 明體
%\newCJKfontfamily{\CR}{cwTeX Q Yuan Medium}		% CwTex 圓體
%-----------------------------------------------------------------------------------------------------------------------
\XeTeXlinebreaklocale "zh"             		%這兩行一定要加,中文才能自動換行
\XeTeXlinebreakskip = 0pt plus 1pt     		%這兩行一定要加,中文才能自動換行
%-----------------------------------------------------------------------------------------------------------------------
\newcommand{\cw}{\texttt{cw}\kern-.6pt\TeX}	% 這是 cwTex 的 logo 文字
\newcommand{\imgdir}{fig/}				% 設定圖檔的目錄位置

%\titleformat{command}[shape]{format}{label}{sep}{before}[after]
%\titleformat{\chapter}[hang]{\fontsize{16}{16}\bfseries\sffamily\filcenter}{\zhnum{chapter}、}{.5em}{}
%\titlespacing*{\chapter}{0pt}{0pt}{16pt}
\titleformat{\section}{\fontsize{12}{12}\bfseries\sffamily}{\zhnum{section}、}{.5em}{}
\titleformat{\subsection}{\fontsize{12}{12}\bfseries\sffamily}{(\zhnum{subsection})}{.5em}{}

\numberwithin{figure}{section} %圖標號含章節
\numberwithin{table}{section}
\numberwithin{equation}{section}
%\renewcommand{\thechapter}{\Alph{chapter}}
\renewcommand{\tablename}{表}					% 改變表格標號文字為中文的「表」(預設為 Table)
\renewcommand{\figurename}{圖}				% 改變圖片標號文字為中文的「圖」(預設為 Figure)
\renewcommand\contentsname{目錄}
\renewcommand\listfigurename{圖目錄}
\renewcommand\listtablename{表目錄}
\renewcommand{\appendixname}{附~錄}

% 設定顏色 see color Table: http://latexcolor.com
\definecolor{slight}{gray}{0.9}				
\definecolor{airforceblue}{rgb}{0.36, 0.54, 0.66} 
\definecolor{arylideyellow}{rgb}{0.91, 0.84, 0.42}
\definecolor{babyblue}{rgb}{0.54, 0.81, 0.94}
\definecolor{cadmiumred}{rgb}{0.89, 0.0, 0.13}
\definecolor{coolblack}{rgb}{0.0, 0.18, 0.39}
\definecolor{beaublue}{rgb}{0.74, 0.83, 0.9}
\definecolor{beige}{rgb}{0.96, 0.96, 0.86}
\definecolor{bisque}{rgb}{1.0, 0.89, 0.77}
\definecolor{gray(x11gray)}{rgb}{0.75, 0.75, 0.75}
\definecolor{limegreen}{rgb}{0.2, 0.8, 0.2}
\definecolor{splashedwhite}{rgb}{1.0, 0.99, 1.0}
\definecolor{carolinablue}{rgb}{0.6, 0.73, 0.89}
\definecolor{lavenderblue}{rgb}{0.8, 0.8, 1.0}
\definecolor{lightblue}{rgb}{0.68, 0.85, 0.9}
\definecolor{wtblue}{rgb}{0.612,0.761,0.898}
\definecolor{wblue}{rgb}{0.871,0.918,0.965}      

%---------------------------------------------------------------------
% 映出程式碼 \begin{lstlisting} 的內部設定
\lstset
{	language=[LaTeX]TeX,
    breaklines=true,
    %basicstyle=\tt\scriptsize,
    basicstyle=\tt\normalsize,
    keywordstyle=\color{blue},
    identifierstyle=\color{black},
    commentstyle=\color{limegreen}\itshape,
    stringstyle=\rmfamily,
    showstringspaces=false,
    %backgroundcolor=\color{splashedwhite},
    backgroundcolor=\color{slight},
    frame=single,							%default frame=none 
    rulecolor=\color{gray(x11gray)},
    framerule=0.4pt,							%expand outward 
    framesep=3pt,							%expand outward
    xleftmargin=3.4pt,						%to make the frame fits in the text area. 
    xrightmargin=3.4pt,						%to make the frame fits in the text area. 
    tabsize=2								%default :8 only influence the lstlisting and lstinline.
}